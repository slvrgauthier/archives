\documentclass[a4paper]{article}
\usepackage[french]{babel}
%\usepackage[T1]{fontenc}
\usepackage[utf8]{inputenc}
\pagestyle{empty}

\topmargin=-2.5cm
\textheight=26cm
\evensidemargin=-1cm
\oddsidemargin=-1cm
\textwidth=18cm

\title{Curriculum Vit\ae}
\author{Clément Roque}
\date{\today}

\begin{document}
\maketitle

\section{\'Etat Civil}
\begin{itemize}
\item Homme
\item Date de naissance 20/06/1992
\item Lieu de naissance: Montpellier
\item Adresse: 445 chemin de Maurin 34430 st jean de védas
\item Numero de telephone: 0648032110
\item Adresse mail: clement.roque@etud.univ-montp2.fr
\end{itemize}

\section{Formation}
\begin{itemize}

\item \'Etudiant en Licence informatique à la Faculté des Sciences Montpellier 2 (2010-2013)
\end{itemize}

 \subsection{Diplômes}
 \begin{itemize}
 \item Baccalauréat Scientifique session 2010
\item Brevet d'Aptitude aux Fonction d'Animateur (BAFA)
\item C2i
\end{itemize}

\subsection{Stages}

\begin{itemize}
\item 2 semaines dans le magasin d'informatique Micro Direct (2007, 2008)
\end{itemize}

\section{Expériences Professionelles}
\begin{itemize}
\item 2 mois en tant qu'animateur pour la jeunesse en AlSH 2012 (3/10ans)
\item  2 semaines en tant qu'animateur pour la jeunesse en colonie (ACM) (14/17ans) 2011
\item  2 semaines en tant qu'animateur pour la jeunesse en camps de ski (11/17ans)  2011/2012
\item  animation 2 vendredi soir par mois d'un groupe de jeune (14/17ans) 2011/2013
\item  2 mois initiation sportive (utlimate freesbee) dans une école du cres 2012

\item  Saison rugby 2011/2013 stadier pour la société ASF
\end{itemize}

\section*{Compétences Informatiques}
\begin{itemize}

\item Programmation imperative
\item Programmation orienté objet
\item Programmation fonctionelle
\item Programmation d'interfaces graphiques
\item Programmation web
\item Aptitudes à l'optimisation d'algorithmes
\item Connaissances en manipulation de Base de Données
\item Maitrise de gestionnaire de version (subversion, google code)
\item Technologies maitrisées (C/C++, java, HTML, scheme, SQL, Python, LaTeX)

\end{itemize}

\section*{Compétences linguistiques}
\begin{itemize}
\item Langue maternelle: Français
\item Niveau d'anglais: courant
\item Espagnol: scolaire
\end{itemize}

\end{document} 
