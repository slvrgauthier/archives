\documentclass{beamer}

%\documentclass[handout]{beamer}
%\usepackage{pgfpages}
%\pgfpagesuselayout{4 on 1}[a4paper, border shrink=5mm, landscape] 

\mode<presentation>
{
  \usetheme{Warsaw}
  \setbeamercovered{transparent}
}

\usepackage[utf8]{inputenc}
\usepackage{times}
\usepackage[T1]{fontenc}
\usepackage[francais]{babel}

\title{Introduction au language HTML}

\author{
Ferrand Anthony 
\and Tran Quang Dung
}


\institute
{
  	L3 Informatique\\
Université Montpellier II
}

\date
{Jeudi 4 Octobre 2012}

\AtBeginSubsection[]
{
 \begin{frame}<beamer>{Table des matières}
    \tableofcontents[currentsection,currentsubsection]
  \end{frame}
}

\begin{document}

\begin{frame}
  \titlepage
\end{frame}

\begin{frame}
\frametitle{Table des matières}
\tableofcontents
\end{frame}

\section{Préambule : Présentation du HTML}

\begin{frame}{Préambule : Présentation du HTML}
\frametitle{Préambule : Présentation du HTML}
  \begin{itemize}
  \item
    Hypertext Markup Language
\pause
  \item
    Langage balisé
\pause
  \item
     Format de données conçu pour représenter les pages web
  \end{itemize}
\end{frame}

\section{HTML, un langage vieux, mais toujours d'actualité}

\subsection[Origine]{L'origine d'HTML}

\begin{frame}
\frametitle{L'origine d'HTML}
  \begin{itemize}
  \item
    1980 : Création du système ENQUIRE par le physicien du CERN Tim Berners-Lee
\pause
  \item
   1991: Premiere description public d'HTML : "HTML Tags"
\pause
  \item
     1993: Premiere proposition d'une spécification de HTML :"HTML Internet-Draft
\pause
\item
    Influencé par SGML
  \end{itemize}
\end{frame}

\subsection[HTML 1.0]{HTML 1.0 : Spécification technique}
\begin{frame}
\frametitle{HTML 1.0 : Spécification technique}
\begin{itemize}
\item
    Navigateur NCSA Mosaic : Revolution du Web
\begin{itemize}
\item
Arrivé des images et des formulaires
\item
Mode interactif : Possibilité de commerce électronique
\end{itemize}
\pause
\item
 1994: Netscape Navigator : le début de l'ere moderne
\begin{itemize}
\item
Attributs de texte, clignotement, centrage
\end{itemize}
\pause
Deux branches de développement
\begin{itemize}
\item
L’impact visuel des pages web
\item
 domaines d’applications et description sémantique
\end{itemize}
  \end{itemize}
\end{frame}

\subsection[HTML 2 à 4]{HTML 2.0, 3.0, 4.0 : les progrès majeurs}
\begin{frame}
\frametitle{HTML 2.0, 3.0, 4.0 : les progrès majeurs}
\begin{itemize}
  \item
1995: RFC 1866 décrivant HTML 2.0 fini sans additions de Netscape Navigator 
\pause
  \item
   14 janvier 1997: Spécification de HTML 3.2 : additions de Netscape Navigator + Internet Explorer
\begin{itemize}
\item
   Standardisation des tables et autres éléments de présentation
\item
  Prévision du support des styles et des scripts.
\end{itemize}
\pause
  \item
     18 décembre 1997: HTML 4.0
\begin{itemize}
\item
  Séparation structure/présentation du document
\item
 Informations supplémentaires des contenus complexes(tableaux,sigles,formulaires)
\item
3 variantes du formats (strict, transitional, frameset)
  \end{itemize}
\end{itemize}
\end{frame}

\subsection[Futur]{HTML un langage tourné vers le futur}

\begin{frame}
\frametitle{HTML un langage tourné vers le futur}
  \begin{itemize}
  \item
      A partir des années 2000, HTML abandonné au profit de XHTML
   \begin{itemize}
  \item
Contestation de XHTML par les fabricants de Navigateur Web
  \end{itemize}
\pause
  \item
      A partir de 2007, XHTML abandonné au profit de HTML 5
  \begin{itemize}
  \item
Travail encadré par Chris Wilson (Microsoft) et par Michael Smith (W3C)
\item
Evolution de la sémantique des éléments (Document, application) présent sur le web
\item
Extention via XML tout en gardant la compatibilité avec les navigateurs actuels
\item
Ajout de "Widget" (barre de progression, menu, ...)
\end{itemize}
\pause
\item
HTML 5 se veut un langage en perpétuelle amélioration
  \end{itemize}
\end{frame}

\section{HTML, un langage simple mais efficace}

\subsection[Principe]{Le principe d'un langage balisé}

\begin{frame}
\frametitle{Le principe d'un langage balisé}
\begin{itemize}
\item
HTML = ensemble d' \texttt{éléments}:
\begin{itemize}
\item
Une paire de \texttt{balises}, l'un entrante, l'autre fermante.
\item
Les \texttt{attributs} dans la balise entrante
\item
Le contenu entre les balises
\end{itemize}
\pause
\item
Et des \texttt{références d'entité} (pour représenter les caractères Unicodes)
\end{itemize}
\end{frame}

\begin{verbatim}
 Exemple HTML
<!DOCTYPE html>
<html>
   <head>
      <title>Hello HTML</title>
   </head>
   <body>
      <p>  Ceci est une phrase </p>
   </body>
</html>
 \end{verbatim}

\begin{frame}
\frametitle{Le principe d'un langage balisé}
\begin{itemize}
\item
Integration du CSS(Cascade Style Sheets):
\begin{itemize}
\item
pour la présentation sémantique(couleur,font,l'alignement...)
\end{itemize}
\end{itemize}
\end{frame}

\begin{verbatim}  
Exemple Css
html {
   font-family: Arial, Helvetica, sans-serif;
   font-size: 95%;
   color: #dbdbdb;
   height : 1880px;
/*width : 1347px;*/
}
body{
   background: url('bg.jpg');
}
h1 {
   margin : 40px 0px 40px 400px;
   font-size : 28px;
   color : #0000ff;
}
\end{verbatim}

\subsection[Extension]{Présentation des extensions}

\begin{frame}
\frametitle{Présentation des extensions}
  \begin{itemize}
  \item
       PHP : l'éditeur dynamique de page Web
  \begin{itemize}
  \item
	Exécuté coté serveur
  \item
	Lien avec les bases de données
\end{itemize}
\pause
\item
JavaScript : le langage Script
  \begin{itemize}
  \item
	Exécuté coté client
  \item
	Orienté objet
\end{itemize}
\pause
\item
       Flash : l'animation sur le Web
  \end{itemize}
\end{frame}

\section{HTML, un langage qui mérite une conclusion}

\subsection[Avantages et inconvénients]{Avantages et inconvénients du langage}
\begin{frame}
\frametitle{Avantages et inconvenants du langage}
  \begin{itemize}
\item
\alert{Avantages}
\begin{itemize}
  \item
       Universelle et en dehors de toute marque, facilité de "recycler" des éléments
\item
	Hypertexte : Permet de gérer de nombreuses informations sans être sauvegardé localement
\end{itemize}
\pause
\item
\alert{Inconvénients }
\begin{itemize}
  \item
       "Lourdeur" du langage : Une page est souvent produits par plusieurs sources
\end{itemize}
  \end{itemize}
\end{frame}

\subsection[Remplaçant?]{Un remplaçant envisageable?}
\begin{frame}
\frametitle{Un remplaçant envisageable?}
\begin{itemize}
\item
Aucune contestation de l'omniprésence du HTML sur le Web
\item
Un avenir serein pour le HTML est à prévoir!
\end{itemize}
\end{frame}

\end{document}