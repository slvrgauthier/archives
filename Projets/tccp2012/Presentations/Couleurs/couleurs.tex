\documentclass{beamer}
\usepackage{beamerthemeshadow}
\usepackage[french]{babel}
\usepackage[T1]{fontenc}
\usepackage[utf8]{inputenc}

\begin{document}
\title{Techniques et utilisation des couleurs}  
\author{CHAMBONNET Kevin - PATIN Nicolas}
\date{\today} 

\frame{\titlepage} 

\frame{\frametitle{Plan}\tableofcontents}


\section{Definitions}
\subsection{Optique}
\frame{\frametitle{Les Couleurs en Optique}
%IMG : Spectre
\includegraphics[scale=0.6]{spectre.jpg}
}

\subsection{Informatique}
\frame{\frametitle{Les Couleurs en Informatique}
En informatique, les couleurs sont définies par deux facteurs :
\begin{itemize}
\item Une \textbf{Base}
\item Une \textbf{Quantification}
\end{itemize}
}

\subsection{Base}
\frame{\frametitle{Les Bases}
Une \textbf{Base} est un ensemble fini d'élements.

\begin{block}{Propriétés}
	\begin{itemize}
	\item Chaque couleur est obtenue par combinaison d'éléments de la base.
	\item Un élément de la base ne peut pas être combinaison d'autres éléments de la base.
	\end{itemize}
\end{block}

}

\subsection{Quantification}
\frame{\frametitle{La Quantification}
Nombre maximum de valeurs que peut prendre un élément d'une base.\newline
\pause
%IMG : Niveau_couleur_octet_par_pixel
\includegraphics[scale=0.3]{octet.jpg}
}

\section{Synthèses}
\frame{\frametitle{Les différentes Synthèses}
Il existe deux types de combinaison de couleurs :
	\begin{itemize}
	\item \textbf{Additive}
	\item \textbf{Soustractive}
	\end{itemize}
}

\subsection{Additive}
\frame{\frametitle{Synthèse Additive}
Exemples : Optique, Écran\newline
%IMG : cercle_addit
\includegraphics[scale=0.4]{cercle_add.png}
}

\subsection{Soustractive}
\frame{\frametitle{Synthèse Soustractive}
Exemples : Peinture, Impression\newline
%IMG : cercle_soustr
\includegraphics[scale=0.4]{cercle_sous.png}
}

\subsection{Couleurs Complémentaires}
\frame{\frametitle{La Complémentaire d'une Couleur}
%IMG : cercle_chrom + Addition_coul
\includegraphics[scale=0.2]{cercle_chro.jpg}
\hspace{2em}
\pause
\includegraphics[scale=0.3]{add_color.png}
}

\section{Modélisations}
\frame{\frametitle{Les différentes Modélisations}
Liste de quelques modélisations :
	\begin{itemize}
	\item \textbf{RVB}
	\item \textbf{TSL}
	\item \textbf{CMJ}
	\item CIE XYZ
	\item CIE Luv
	\item ...
	\end{itemize}
}

\subsection{RVB}
\frame{\frametitle{Rouge - Vert - Bleu}
Type de Combinaison : Additive
\pause
\newline
Un pixel est composé de 3 sous-pixels
%IMG : pixel_Zooom
\includegraphics[scale=0.3]{pixel.png}
}

\subsection{TSL}
\frame{\frametitle{Teinte - Saturation - Luminance}
Base : 256 couleurs
\pause
\newline
%IMG : Lum
\includegraphics[scale=0.4]{teinte.jpg}
\pause
%IMG : Sat
\includegraphics[scale=0.4]{TSL_resume.jpg}
}

\subsection{CMJ}
\frame{\frametitle{Cyan - Magenta - Jaune}
Type de Combinaison : Soustractive\newline
\pause
Utilisé pour les impressions (CMJ-N).
\pause
%IMG : Impr_Zooooooom || Sequence superpos
\includegraphics[scale=0.3]{superpos.jpg}
}

\frame{\frametitle{Cyan - Magenta - Jaune}
%IMG : Impr_Zooooooom || Sequence superpos
\includegraphics[scale=0.6]{rvb_vs_cmjn.jpg}
}

\frame{\frametitle{Cyan - Magenta - Jaune}
Passage du RVB -> CMJ :\newline
C = 1 - (R / 255) \newline
M = 1 - (V / 255) \newline
J = 1 - (B / 255) \newline

min = minimum(C, M, J) \newline

C = (C - min) / (1 - min) \newline
M = (M - min) / (1 - min) \newline
J = (J - min) / (1 - min) \newline

}


\section{Sources}
\frame{\frametitle{Sources}
\begin{itemize}
\item http://www.fr.wikiversity.org/wiki/Colorimétrie/\newline Synthèse\_Additive\_et\_soustractive\_des\_couleurs
\item http://www.colorimetrie.be
\end{itemize}
}

\end{document}

