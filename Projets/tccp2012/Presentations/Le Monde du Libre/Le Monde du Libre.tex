\documentclass{beamer}
\setbeamercovered{transparent}

\usepackage{lmodern}                %meilleure police
\usepackage[utf8]{inputenc}
\usepackage[T1]{fontenc}
\usepackage[francais]{babel}
\usepackage{graphicx} 

\usepackage{hyperref}

\usepackage{tikz}
\usepackage{pgfplots}
\usetikzlibrary{plotmarks}

%\usetheme[secheader]{Madrid}    % titre en haut
 \usetheme{Marburg} 
 
%\useoutertheme{smoothbars}       %ligne d'en-tête et de pied de page
\setbeamertemplate{caption}[numbered] % pour numéroter tables et figures !

\institute{UM2}
\date{Univ. Montpellier 2}
\title{Le Monde du Libre}
\author{PEREIRA Julien \and MASIP Florian}


\begin{document}

%-diapo 1---------------------------------------------
\begin{frame}
	\titlepage
\end{frame}

%-diapo 2---------------------------------------------
\begin{frame}
	\frametitle{Sommaire}
	\tableofcontents[hideothersubsections]
\end{frame}

%-diapo 3---------------------------------------------
\section{Introduction}

\begin{frame}
	\frametitle{Introduction}
	Le monde de l'informatique est un vaste domaine, dans lequel différentes \textbf{philosophies} s'affrontent.
	\newline \newline
	De nos jours, le nombre de logiciels accessibles à n'importe quel utilisateur ne cesse d'augmenter et l'utilisateur peut lui même s'impliquer dans sa conception.
	\newline\newline 
	Les logiciels libres permettent à tout utilisateur de \textbf{personnaliser} son environnement à sa guise.
	\newline\newline
	Le monde du libre est défendu par deux mouvements 
\end{frame}

%-diapo 4---------------------------------------------
\section{Historique}
\subsection[Logiciel libre]{Logiciel libre}

\begin{frame}
	\frametitle{Logiciel libre}
	\begin{itemize}
		\item Notion décrite pour la première fois dans la première moitié des années 1980 par \textbf{Richard Stallman.}
		\newline
		\begin{itemize}
			\item Formalisée et popularisée avec le projet \textbf{GNU} et la \textbf{Free Software Foundation} (FSF)
			\newline
		\end {itemize}
		\begin{figure}
				\centering
				\includegraphics<1->[width=2cm]{GNU.png}\includegraphics<1->[width=4cm]{FSF.png}
		\end{figure}
		\pause
		
		\item \textbf{Free Software Foundation}: met en avant la finalité philosophique et politique de la licence
		\newline
		\begin{itemize}
			\item le logiciel non libre est un problème social et le logiciel libre en est la solution.
		\end{itemize}
	\end{itemize}
\end{frame}


%-diapo 5-------------------------------------------
\subsection[open source]{open source}

\begin{frame}
	\frametitle{open source}
	\begin{itemize}
		\item La désignation open source s'applique aux logiciels dont la licence respecte des critères précisément établis par\newline
		l' \textbf{Open Source Initiative} (OSI)
	 	\newline
	 	\begin{itemize}
	 		\item crée par \textbf{Eric Steven}  et \textbf{Bruce Perens} 
	 		\item délivre le label \textbf{OSI approved} aux licences 
	 		\newline
		\end{itemize}
		\begin{figure}
				\centering
				\includegraphics<1->[width=2cm]{OSI.png}
		\end{figure}
		\pause
		
		\item \textbf{Open Source Definition}: met l'accent sur la méthode de développement et de diffusion du logiciel.
		\newline
		\begin{itemize}
			\item un logiciel non libre est une solution sous-optimale
		\end{itemize}
	\end{itemize}
\end{frame}


%-diapo 6---------------------------------------------
\subsection[Points commun / distinctions]{Points commun / distinctions}

\begin{frame}
	\frametitle{Points commun / distinctions}
	\begin{itemize}
	\item La différence fondamentale entre les deux mouvements se situe dans \textbf{leurs valeurs et leurs façons de voir le monde}
	\newline\newline
	\item Le mouvement du logiciel libre et l'open source sont aujourd'hui des mouvements séparés, avec \textbf{des idées\newline et des objectifs différents}
	\newline\newline
	\item Les définitions officielles du logiciel libre et de l'open source \textbf{renvoient dans la pratique aux mêmes licences}
	\end{itemize}
\end{frame}


%-diapo 7---------------------------------------------
\section[Les Licences et l'Open source]{Les Licences et l'Open source}
\subsection[Les Licences]{Les Licences}

\begin{frame}
	\frametitle{Les Licences}
	Pour classifier les différentes utilisations qu'il est possible de faire d'une oeuvre, il à était créé des licences possédant chacune des régles définies par différentes organisations :
	\pause
	\begin{itemize}
		\item Pour protéger les droits d'auteur et la liberté de l'utilisateur.
		\pause
		\item Les licences sont commune à l'\textbf{ensemble} des "oeuvres de l'esprit" (i.e musiques, peintures ... mais aussi pour l'informatique).
		\pause
		\item Elles sont plus au moins restrictives pour l'utilisateur et plus ou moins lucratives pour l'auteur.
		\pause
		\item Elles sont représentées par des symboles : \includegraphics<5->[width=1cm]{symbNonComm.png}  \includegraphics<5->[width=1cm]{symbCopyleft.png} 
	\end{itemize}
\end{frame}

%-diapo 8---------------------------------------------
\subsection[Les catégories]{Les catégories}

\begin{frame}
	\frametitle{Les catégories}
	\begin{itemize}
		\item Les licences sont regroupées en plusieurs domaines : 
		\pause
		\begin{itemize}
			\item\footnotesize Les licences libres
			\pause
			\begin{itemize}
				\item\footnotesize \underline{Les licences libres non-copyleft} : Permettent la réutilisation de tout ou d'une partie du logiciel sans restriction.
				\item\footnotesize \underline{Les licences libres copyleft} : Permettent la réutilisation, l'étude, la modification du tout ou d'une partie du logiciel sans restriction \textbf{dans la mesure où cette autorisation reste préservée}
			\end{itemize}
			\pause
			\item\footnotesize \underline{Les licences libres de diffusion} : Ont le même principe que les licences libres mais favorisent la diffusion du programme.
			\pause
			\item\footnotesize \underline{Les licences propriétaires} : Tout programme ne respectant pas une des conditions des licences libres.
		\end{itemize}
	\end{itemize}
	\pause
	Les logiciels n'ayant pas de licence appartiennent au \textbf{domaine public}, c'est à dire pas protégés par les droits d'auteur.
\end{frame}

%-diapo 9---------------------------------------------
\subsection[L'Open source]{L'Open source}

\begin{frame}
	\frametitle{L'Open source}
	L'open source s'applique aux logiciels dont la licence respecte des critères précisément établis par l'Open Source Initiative.
	\pause
	\begin{itemize}
		\item\footnotesize Il s'attache aux avantages d'une méthode de développement au travers de la réutilisation du code source. 
		\pause
		\item\footnotesize L'open source s'applique aux logiciels mais aussi aux \textbf{travaux dérivés} \textit{(algorithmes, fonctions ...)}.
		\pause
		\item\footnotesize Souvent, un logiciel libre est qualifié d'open source, car les licences compatibles open source englobent les licences libres (selon la définition de la Free Software Foundation).
		\pause
		\item\footnotesize Le terme open source est en concurrence avec le terme \textit{free software} recommandé par la Free Software Foundation.\\
		Attention : Pourtant les termes n'ont pas les mêmes critéres (exemple le projet \textit{Darwin} de Apple certifié open source par OSI mais pas par FSF).
	\end{itemize}
\end{frame}

%-diapo 10---------------------------------------------
\section[Les logiciels libres]{Les logiciels libres}
\subsection[Définition]{Définition}

\begin{frame}
	\frametitle{Les logiciels libres}
	Les logiciels libres appartiennent soit au domaine public soit aux licences libres. Ils doivent conférer à l'utilisateur quatres libertés définies par la Free Software Fondation :
	\begin{enumerate}
		\pause
		\item\footnotesize La liberté d'exécuter le programme, pour tous les usages.
		\pause
		\item\footnotesize La liberté d'étudier le fonctionnement du programme et de l'adapter à ses besoins.
		\pause
		\item\footnotesize La liberté de redistribuer des copies du programme (ce qui implique la possibilité aussi bien de donner que de vendre des copies).
		\pause
		\item\footnotesize La liberté d'améliorer le programme et de distribuer ces améliorations au public, pour en faire profiter toute la communauté.
	\end{enumerate}
	\pause
	Ces régles sont irrévocable et la redistribution du programme peut se faire sous toute forme à la condition de toujours rendre disponible le code source correspondant.
\end{frame}

%-diapo 10---------------------------------------------
\subsection[Exemple de logiciels libres]{Exemple de logiciels libres}

\begin{frame}
	\frametitle{Exemple de logiciels libres}
	Les logiciels libres sont omniprésents dans l'informatique :
	\begin{itemize}
		\pause
		\item Noyau du système d'exploitation LINUX, \textbf{GNU/Linux} : \textit{Licence GNU GPL}.
		\pause
		\item Environnements graphiques de bureau \textbf{GNOME} et \textbf{KDE} : \textit{Licence GNU GPL}.
		\pause
		\item Navigateur internet \textbf{Chromium} (google chrome) : Majoritairement \textit{licence BSD}.
		\pause
		\item Serveur de gestion de nom de domaine \textbf{BIND} : \textit{Licence ISC}.
	\end{itemize}
\end{frame}

%-diapo 11---------------------------------------------
\section[Avantages/Inconvéniants]{Avantage, inconvéniant et usage du Libre}
\subsection[Avantages]{Avantages}

\begin{frame}
	\frametitle{Avantages}
	Les logiciels libres possédent certains avantages sur les logiciels propriétaires :
	\begin{itemize}
		\pause
		\item Une communauté : Un support constant de \textbf{bénévoles} dans le monde pour localiser tout problème dans le code.
		\pause
		\item La personnalisation du programme grâce à la \textbf{modification du code source} aussi bien fonctionnellement et visuellement.
		\pause
		\item Le logiciel libre tend vers la gratuité même si elle n'est pas obligatoire.
		\pause
		\item Ils utilisent un format ouvert menant vers une meilleure compatibilité.
		\pause
		\item En tant que développeur, il est plus facile de ce faire aider par la communauté en montrant son code source.
	\end{itemize}
\end{frame}
%-diapo 12--------------------------------------------
\subsection[Inconvéniants]{Inconvéniants}

\begin{frame}
	\frametitle{Inconvéniants}
	Mais également des inconvéniants :
	\begin{itemize}
		\pause
		\item Le suivi du logiciel est au bon vouloir de la communauté et/ou des sponsors.
		\pause
		\item Demande des connaissances pour utiliser au mieux le logiciel.
		\pause
		\item Le code source étant disponible, la sécurité par l'\textbf{obscurité} n'est pas possible.
	\end{itemize}
\end{frame}

%-diapo 13---------------------------------------------
\section[L'avenir du Libre]{L'avenir du Libre}
\subsection[L'évolution]{L'évolution}

\begin{frame}
	\frametitle{L'avenir du Libre}
	Le logiciel libre s'impose de plus en plus comme une solution de remplacement aux logiciels proprétaires.
	
	\begin{itemize}
		\pause
		\item Le marché du logiciel libre est très largement \textbf{orienté service}.
		\pause
		\item \textbf{Solution moins couteuse} : choix du gouvernement français actuel et des universités.
		\pause
		\item Les entreprises ne \textbf{dépendent} plus de l'éditeur du logiciel et des lois qu'il impose.
		\pause
		\item \textbf{Standardisation} des composants technologiques, \textbf{interopérabilité} entre les différents produits.
		\pause
		\item Le logiciel produit par l'entreprise peut être diffusé ou même revendu après modifications.
	\end{itemize}	
\end{frame}

%-diapo 13---------------------------------------------
\subsection[Le financement]{Le financement}

\begin{frame}
	\frametitle{Le financement}
	Les entreprises open source ont plusieurs modèles économiques :\newline
	\pause
	\begin{itemize}
		\item\footnotesize \textbf{L'entraide de la communauté}, dons et/ou participations bévévoles (Mozilla Firefox).
		\pause
		\item\footnotesize \textbf{Facturation de services} inclus dans un système libre (Ubuntu One) ou autour du logiciel (installation, formation).
		\pause
		\item\footnotesize \textbf{Le financement par les grands acteurs de l'industrie de l'informatique} (Linux financé entre autres par RedHat et IBM).
		\pause
		\item\footnotesize Le paiment de \textbf{royalties} (approche open source tronquée).
		\pause
		\item\footnotesize \textbf{Associations} entre les fabricants de matériels informatiques et les éditeurs pour une meilleure compatibilité.
		\newline
	\end{itemize}
	\pause
	Ce système de financement reste moins stable que celui des logiciels propriétaires.
\end{frame}

%-diapo 14---------------------------------------------
\subsection[Quelques chiffres]{Quelques chiffres}

\begin{frame}
	\frametitle{Quelques chiffres}
	\begin{tikzpicture}
		\begin{axis}[xlabel={Année}, ylabel={CA en millions}, title={Evolution du CA des logiciels libres}, enlarge x limits=false, enlarge y limits=false, x tick label style={/pgf/number format/1000 sep=}, y tick label style={/pgf/number format/1000 sep=}]
			\addplot[ybar, ybar interval=0] coordinates{
			(2002,60) (2003,100) (2004,140) (2005,250) (2006,440) (2007,733) (2008,1100) (2009,1500) (2010,2200) (2011,2501) (2012,3000)};
		\end{axis}
	\end{tikzpicture}
\end{frame}

%-diapo 14---------------------------------------------
\section[Conclusion]{Conclusion}

\begin{frame}
	\frametitle{Conclusion}
	Le monde du libre est un domaine très vaste, sa classification en Licence permet de regrouper les logiciels ayant les mêmes caractéristiques tout en protégeant le droit d'auteur.\newline
	\begin{itemize}
		\item Pour les utilisateurs, les logiciels libres représentent une trés bonne alternative au logiciel propriétaire, bien souvent payant et restrictif.\newline
		\item Pour les développeurs, le choix du libre n'est pas évident. Le financement necessaire est bien plus difficile à trouver.	
	\end{itemize}
\end{frame}

%-diapo 14---------------------------------------------
\section[Sources]{Source}

\begin{frame}
	\frametitle{Sources}
	\begin{itemize}
		\item \url{http://www.gnu.org}
		\item \url{http://www.wikipedia.fr}
		\item \url{http://www.fsf.org}
		\item \url{http://opensource.org}
	\end{itemize}
\end{frame}


\end{document}
